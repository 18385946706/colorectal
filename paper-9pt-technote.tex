% This template has been tested with IEEEtran of 2015.

% !TeX spellcheck = en-US
% !TeX encoding = utf8
% !TeX program = pdflatex
% !BIB program = bibtex
% -*- coding:utf-8 mod:LaTeX -*-

%cmap has to be loaded before any font package (such as newtxmath)
\RequirePackage{cmap}

% DO NOT DOWNLOAD IEEEtran.cls - Use the one of your LaTeX distribution
\documentclass[9pt,technote]{IEEEtran}[2015/08/26]

% Balance the last page
% The pbalance package (see https://ctan.org/pkg/pbalance) "just works" (in contrast to balance.sty or other solutions)
\usepackage{pbalance}

% use nicer font for code
\usepackage[zerostyle=b,scaled=.75]{newtxtt}

% for demonstration purposes
\usepackage{mwe}

\usepackage[T1]{fontenc}
\usepackage[utf8]{inputenc} %support umlauts in the input

\usepackage{graphicx}

%Set English as language and allow to write hyphenated"=words
\usepackage[ngerman,main=english]{babel}
%Hint by http://tex.stackexchange.com/a/321066/9075 -> enable "= as dashes
\addto\extrasenglish{\languageshorthands{ngerman}\useshorthands{"}}

% backticks (`) are rendered as such in verbatim environment. See https://tex.stackexchange.com/a/341057/9075 for details.
\usepackage{upquote}

%extended enumerate, such as \begin{compactenum}
\usepackage{paralist}

%for easy quotations: \enquote{text}
\usepackage{csquotes}

%enable margin kerning
\RequirePackage{iftex}
\ifPDFTeX
  \RequirePackage[%
    final,%
    expansion=alltext,%
    protrusion=alltext-nott]{microtype}%
\else
  \RequirePackage[%
    final,%
    protrusion=alltext-nott]{microtype}%
\fi%
% \texttt{test -- test} keeps the "--" as "--" (and does not convert it to an en dash)
\DisableLigatures{encoding = T1, family = tt* }

%tweak \url{...}
\usepackage{url}
%\urlstyle{same}
%improve wrapping of URLs - hint by http://tex.stackexchange.com/a/10419/9075
\makeatletter
\g@addto@macro{\UrlBreaks}{\UrlOrds}
\makeatother
%nicer // - solution by http://tex.stackexchange.com/a/98470/9075
%DO NOT ACTIVATE -> prevents line breaks
%\makeatletter
%\def\Url@twoslashes{\mathchar`\/\@ifnextchar/{\kern-.2em}{}}
%\g@addto@macro\UrlSpecials{\do\/{\Url@twoslashes}}
%\makeatother

% Diagonal lines in a table - http://tex.stackexchange.com/questions/17745/diagonal-lines-in-table-cell
% Slashbox is not available in texlive (due to licensing) and also gives bad results. This, we use diagbox
%\usepackage{diagbox}

\usepackage{booktabs}

% Required for package pdfcomment later
\usepackage{xcolor}

% For listings
\usepackage{listings}
\lstset{%
  basicstyle=\ttfamily,%
  columns=fixed,%
  basewidth=.5em,%
  xleftmargin=0.5cm,%
  captionpos=b}%

% Enable nice comments
\usepackage{pdfcomment}
%
\newcommand{\commentontext}[2]{\colorbox{yellow!60}{#1}\pdfcomment[color={0.234 0.867 0.211},hoffset=-6pt,voffset=10pt,opacity=0.5]{#2}}
\newcommand{\commentatside}[1]{\pdfcomment[color={0.045 0.278 0.643},icon=Note]{#1}}
%
% Compatibility with packages todo, easy-todo, todonotes
\newcommand{\todo}[1]{\commentatside{#1}}
% Compatiblity with package fixmetodonotes
\newcommand{\TODO}[1]{\commentatside{#1}}

% Bibliopgraphy enhancements
%  - enable \cite[prenote][]{ref}
%  - enable \cite{ref1,ref2}
% Alternative: \usepackage{cite}, which enables \cite{ref1, ref2} only (otherwise: Error message: "White space in argument")
%
% Doc: http://texdoc.net/natbib
\ifCLASSOPTIONcompsoc
  % IEEE Computer Society needs nocompress option at cite.sty
  % natbib includes the same functionality
  \usepackage[%
    square,        % for square brackets
    comma,         % use commas as separators
    numbers,       % for numerical citations;
    sort           % orders multiple citations into the sequence in which they appear in the list of references;
    %sort&compress % as sort but in addition multiple numerical citations
                   % are compressed if possible (as 3-6, 15);
  ]{natbib}
\else
  % normal IEEE
  \usepackage[%
    square,        % for square brackets
    comma,         % use commas as separators
    numbers,       % for numerical citations;
    %sort           % orders multiple citations into the sequence in which they appear in the list of references;
    sort&compress % as sort but in addition multiple numerical citations
                   % are compressed if possible (as 3-6, 15);
  ]{natbib}
\fi
% Same fontsize as without natbib
\renewcommand{\bibfont}{\normalfont\footnotesize}
% Enable hyperlinked author names in the case of \citet
% Source: https://tex.stackexchange.com/a/76075/9075
\usepackage{etoolbox}
\makeatletter
\patchcmd{\NAT@test}{\else \NAT@nm}{\else \NAT@hyper@{\NAT@nm}}{}{}
\makeatother

% Enable that parameters of \cref{}, \ref{}, \cite{}, ... are linked so that a reader can click on the number an jump to the target in the document
\usepackage{hyperref}
% Enable hyperref without colors and without bookmarks
\hypersetup{hidelinks,
  colorlinks=true,
  allcolors=black,
  pdfstartview=Fit,
  breaklinks=true}
%
% Enable correct jumping to figures when referencing
\usepackage[all]{hypcap}

%enable \cref{...} and \Cref{...} instead of \ref: Type of reference included in the link
\usepackage[capitalise,nameinlink]{cleveref}
\crefname{lstlisting}{\lstlistingname}{\lstlistingname}
\Crefname{lstlisting}{Listing}{Listings}

%Following definitions are outside of IfPackageLoaded; inside, they are not visible
%
%Intermediate solution for hyperlinked refs. See https://tex.stackexchange.com/q/132420/9075 for more information.
\newcommand{\Vlabel}[1]{\label[line]{#1}\hypertarget{#1}{}}
\newcommand{\lref}[1]{\hyperlink{#1}{\FancyVerbLineautorefname~\ref*{#1}}}

\newenvironment{listing}[1][htbp!]{\begin{figure}[#1]}{\end{figure}}
\newcounter{listing}

\usepackage{xspace}
%\newcommand{\eg}{e.\,g.\xspace}
%\newcommand{\ie}{i.\,e.\xspace}
\newcommand{\eg}{e.\,g.,\ }
\newcommand{\ie}{i.\,e.,\ }

%introduce \powerset - hint by http://matheplanet.com/matheplanet/nuke/html/viewtopic.php?topic=136492&post_id=997377
\DeclareFontFamily{U}{MnSymbolC}{}
\DeclareSymbolFont{MnSyC}{U}{MnSymbolC}{m}{n}
\DeclareFontShape{U}{MnSymbolC}{m}{n}{
  <-6>    MnSymbolC5
  <6-7>   MnSymbolC6
  <7-8>   MnSymbolC7
  <8-9>   MnSymbolC8
  <9-10>  MnSymbolC9
  <10-12> MnSymbolC10
  <12->   MnSymbolC12%
}{}
\DeclareMathSymbol{\powerset}{\mathord}{MnSyC}{180}

% *** SUBFIGURE PACKAGES ***
\ifCLASSOPTIONcompsoc
  \usepackage[caption=false,font=footnotesize,labelfont=sf,textfont=sf]{subfig}
\else
  \usepackage[caption=false,font=footnotesize]{subfig}
\fi

\usepackage{stfloats}

% correct bad hyphenation here
\hyphenation{op-tical net-works semi-conduc-tor}






\begin{document}
%\IEEEoverridecommandlockouts

\title{Quick start for LaTeXing with IEEEtran.cls for\\ IEEE Conferences}

\author{%
  \IEEEauthorblockN{Oliver Kopp}
  \IEEEauthorblockA{University of Stuttgart, Germany\\
    \{lastname\}@ipvs.uni-stuttgart.de}
  \and
  \IEEEauthorblockN{Michael Shell}
  \IEEEauthorblockA{School of Electrical and\\Computer Engineering\\
    Georgia Institute of Technology\\
    Atlanta, Georgia 30332--0250\\
    \url{http://www.michaelshell.org/contact.html}}
}

% use for special paper notices
%\IEEEspecialpapernotice{(Invited Paper)}

% make the title area
\maketitle

% In case you want to add a copyright statement.
%
% Source: https://tex.stackexchange.com/a/200330/9075
%
% All possible solutions:
%  - https://tex.stackexchange.com/a/325013/9075
%  - https://tex.stackexchange.com/a/279134/9075
%  - https://tex.stackexchange.com/q/279789/9075 (TikZ)
%  - https://tex.stackexchange.com/a/200330/9075 - for non-compsocc papers
\iffalse
    \makeatletter
    \def\ps@IEEEtitlepagestyle{%
      \def\@oddfoot{\mycopyrightnotice}%
      \def\@evenfoot{}%
    }
    \makeatother
    \def\mycopyrightnotice{%
      \begin{minipage}{\textwidth}
        \footnotesize
        1551-3203 \copyright 2015 IEEE.
        Personal use is permitted, but republication/redistribution requires IEEE permission.
        \\
        See \url{https://www.ieee.org/publications_standards/publications/rights/index.html} for more information.
      \end{minipage}
      \gdef\mycopyrightnotice{}% just in case
    }
\fi

\begin{abstract}
  \blindtext
\end{abstract}

% For peer review papers, you can put extra information on the cover
% page as needed:
% \ifCLASSOPTIONpeerreview
% \begin{center} \bfseries EDICS Category: 3-BBND \end{center}
% \fi
%
% For peerreview papers, this IEEEtran command inserts a page break and
% creates the second title. It will be ignored for other modes.
\IEEEpeerreviewmaketitle



\section{Introduction}
\label{sec:intro}

\lipsum[1-3]

The remainder of the paper starts with a presentation of related work (\cref{sec:relatedwork}).
It is followed by a presentation of hints on \LaTeX{} (\cref{sec:hints}).
Based on that, we present some Lorem Ipsum (\cref{sec:loremipsum}).
Finally, a conclusion is drawn and outlook on future work is made (\cref{sec:outlook}).

\section{Related Work}
\label{sec:relatedwork}

Winery~\cite{Winery} is a graphical \commentontext{modeling}{modeling with one \enquote{l}, because of AE} tool.
The whole idea of TOSCA is explained by \citet{Binz2009}.

\blindtext[5]

\section{LaTeX Hints}
\label{sec:hints}

\Cref{L1,L2} show listings typeset using the \texttt{lstlisting} environment.

\begin{lstlisting}[
  % one can adjust spacing here if required
  % aboveskip=2.5\baselineskip,
  % belowskip=-.8\baselineskip,
  caption={Example Java Listing},
  label=L1,
  language=Java,
  float]
public class Hello {
    public static void main (String[] args) {
        System.out.println("Hello World!");
    }
}
\end{lstlisting}

\begin{lstlisting}[
  % one can adjust spacing here if required
  % aboveskip=2.5\baselineskip,
  % belowskip=-.8\baselineskip,
  caption={Example XML Listing},
  label=L2,
  language=XML,
  float]
<example attr="demo">
  text content
</example>
\end{lstlisting}

\begin{figure}
  \includegraphics[width=.5\textwidth]{example-grid-100x100bp}
  \caption{Simple Figure. \cite[based on][]{mwe}}
  \label{fig:simple}
\end{figure}

\begin{figure*}
  \centering
  \includegraphics[width=.6\textwidth]{example-image-16x9}
  \caption{16x9 Figure}
  \label{fig:16x9}
\end{figure*}

\begin{figure*}[!b]
  \centering
  \subfloat[Case I]{\includegraphics[width=.4\textwidth]{example-image-a}%
    \label{fig_first_case}}
  \hfil
  \subfloat[Case II]{\includegraphics[width=.4\textwidth]{example-image-b}%
    \label{fig_second_case}}
  \caption{Simulation results for the network.}
  \label{fig_sim}
  % Note that often IEEE papers with subfigures do not employ subfigure
  % captions (using the optional argument to \subfloat[]), but instead will
  % reference/describe all of them (a), (b), etc., within the main caption.
  % Be aware that for subfig.sty to generate the (a), (b), etc., subfigure
  % labels, the optional argument to \subfloat must be present. If a
  % subcaption is not desired, just leave its contents blank,
  % e.g., \subfloat[].
\end{figure*}

\begin{table}
  \caption{Simple Table}
  \label{tab:simple}
  \centering
  \begin{tabular}{ll}
    \toprule
    Heading1 & Heading2 \\
    \midrule
    One      & Two      \\
    Thee     & Four     \\
    \bottomrule
  \end{tabular}
\end{table}

cref Demonstration: Cref at beginning of sentence, cref in all other cases.

\Cref{fig:simple} shows a simple fact, although \cref{fig:simple} could also show something else.
\Cref{fig:16x9} shows an 16x9 image spanning two columns.
\Cref{fig_first_case} is the first subfloat, whereas \Cref{fig_second_case} is the second one.

\Cref{tab:simple} shows a simple fact, although \cref{tab:simple} could also show something else.

\Cref{sec:intro} shows a simple fact, although \cref{sec:intro} could also show something else.

Brackets work as designed:
<test>
One can also input backquotes in verbatim text: \verb|`test`|.

The symbol for powerset is now correct: $\powerset$ and not a Weierstrass p ($\wp$).

\begin{inparaenum}
  \item All these items...
  \item ...appear in one line
  \item This is enabled by the paralist package.
\end{inparaenum}

``something in quotes'' using plain tex or use \enquote{the enquote command}.

You can now write words containing hyphens which are hyphenated (application"=specific) at other places.
This is enabled by an additional configuration of the babel package.
In case you write \enquote{application-specific}, then the word will only be hyphenated at the dash.

\section{Lorem ipsum}
\label{sec:loremipsum}
\lipsum[1-4]

\section{Conclusion and Outlook}
\label{sec:outlook}

\blindtext

% use section* for acknowledgment
\ifCLASSOPTIONcompsoc
  % The Computer Society usually uses the plural form
  \section*{Acknowledgments}
\else
  % regular IEEE prefers the singular form
  \section*{Acknowledgment}
\fi

This work received support from the \href{https://latextemplates.github.io/IEEE/}{IEEE LaTeX Template}.


In the bibliography, use \texttt{\textbackslash textsuperscript} for ``st'', ``nd'', \ldots:
E.g., \enquote{The 2\textsuperscript{nd} conference on examples}.
When you use \href{https://www.jabref.org}{JabRef}, you can use the clean up command to achieve that.
See \url{https://help.jabref.org/en/CleanupEntries} for an overview of the cleanup functionality.

% trigger a \newpage just before the given reference
% number - used to balance the columns on the last page
% adjust value as needed - may need to be readjusted if
% the document is modified later
%\IEEEtriggeratref{8}
% The "triggered" command can be changed if desired:
%\IEEEtriggercmd{\enlargethispage{-5in}}

% Enable to reduce spacing between bibitems (source: https://tex.stackexchange.com/a/25774)
% \def\IEEEbibitemsep{0pt plus .5pt}

\bibliographystyle{IEEEtranN} % IEEEtranN is the natbib compatible bst file
% argument is your BibTeX string definitions and bibliography database(s)
\bibliography{paper}

\ \\ % empty line after bibliogpraphy and that statement
All links were followed on April 18, 2018.

\end{document}
